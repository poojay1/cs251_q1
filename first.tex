\documentclass[12pt, a4paper]{article}
\usepackage[utf8]{inputenc}
\usepackage{graphicx}
\graphicspath{ {/home/pooja/Downloads/} }
\usepackage{amsmath, amssymb}
\usepackage{amsfonts}
\usepackage{amsthm}
\usepackage{algorithm}
\usepackage{algpseudocode}
\usepackage{tabularx}
\usepackage{hyperref}
\usepackage[english]{babel}
\usepackage{pdfpages}
\title{Multiplication Hacks and Tricks}
\author{Pooja Yadav}
\date{6th March, 2017}

\begin{document}
\begin{titlepage}
\maketitle
\vspace{4cm}
\centering
Course Title: Computing Laboratory (CS251)\\
Course Instructor: Dr. Arnab Bhattacharya
\end{titlepage}
\section{Introduction}
Are you looking for tricks that can make arithmetic calculations easier for you. The following document discusses some very simple and easy to remember mathematical tricks that can come in handy when multiplying two digit numbers.
\section{Trick number 1}
\label{sec1}
Suppose you have to multiply two natural numbers between 1 and 100 whose first digits are the same and the sum of their second digit is 10. Multiplying them in the conventional way takes up time and space, and there exists a very simple trick for this problem. By this trick, the multiplication of two 2-digit numbers can be reduced to multiplication of single digit numbers and additions involving one digit.
\subsection{Algorithm}
\begin{algorithm}[H]
input $n_1, n_2$
\begin{algorithmic}[1]
\If {$(n_1<0)$ or $(n_1>100)$ or $(n_2<0)$ or $(n_2>100)$}
\State Exit
\EndIf
\If {($n_1/10$ != $n_2/10$) or (($n_1$ mod 10) + ($n_2$ mod 10) != 10)}
\State Exit
\EndIf
\\
$d_1$ $\gets$ $n_1/10$ \\ 
$d_2$ $\gets$ $n_1$ mod 10 \\
prod $\gets$ $(100*d_1*(d_1+1)) + (d_2 * (10 - d_2))$ \\
\end{algorithmic}
\end{algorithm}
\section{Conclusion}
	The tricks discussed in this article turns out to be useful, but only in special cases like when the first of the 2-digit number is same and the sum of the other digit is 10 \ref{sec1} or the last digit is same and the sum of first digit is 10.\\
	The idea has been generalised to the multiplication of any two n-digit natural numbers in this wikipedia article \url{https://en.wikibooks.org/wiki/Vedic_Mathematics/Techniques/Multiplication}. These concepts have been known since Vedic times and have been found in various Indian manuscripts \cite{Book2}. Deriving ideas Vedic literature, many Indian authors in recent times have published books on these concepts \cite{Book1} which makes this knowledge accessible to the modern day people. 
\section{Bibliography}
\bibliographystyle{unsrt}
\bibliography{ref}
\end{document}

